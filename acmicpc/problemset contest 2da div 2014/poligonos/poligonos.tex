\problem{El juego de los polígonos}{poligonos}

Dos jugadores participan en el juego de los polígonos.  Un polígono convexo con n vértices dividido por n-3 diagonales en n-2 triángulos es necesario. Estas diagonales solo pueden cruzarse en vértices del polígono. Uno de los triángulos es negro y las restantes son de color blanco. Los jugadores avanzan en los giros alternos. Cada jugador, cuando llegue su turno, corta solo un triángulo a partir del polígono. Los jugadores están autorizados a cortar triángulos a lo largo de las diagonales dadas. El ganador es el jugador que corta el triángulo negro.
Su tarea es verificar si el jugador q tiene el primer turno es el ganador.
NOTA: Nosotros llamamos un polígono convexo si un segmento que une dos puntos cualesquiera del polígono se encuentra en el polígono.

\subsection*{Input}

La primera línea de la entrada estándar contiene un entero n, $4 \leq n \leq 50000$. Este es el número de vértices en el polígono. Los vértices del polígono se numeran, las agujas del reloj, a partir 0 de n-1. Las siguientes n-2 líneas contienen descripciones de los triángulos en el polígono. En cada línea hay tres números enteros no negativos a, b, c separados por espacios que son los números de los vértices del i-esimo triángulo. El primer triángulo en una secuencia es negro.

\subsection*{Output}

La salida estándar debe tener una línea con la palabra:

\begin{itemize}
  \item SI, si el jugador que comienza el juego es el que gana,
  \item N0, si no gana.
\end{itemize}

\datos
