\problem{Habilidad Mental}{habilidad}

Existen muchos tests de habilidad mental, uno de ellos fue ideado por un estudiante de la UMSA, consiste en una pantalla llena de puntos de varios colores, debes encontrar mentalmente los dos puntos del mismo color que esten mas cerca el uno del otro. Obviamente a mas colores, mas se tarda en encontrar todos estos puntos, ademas es un buen ejercicio para poder separar lo que se ve y ejercitar el cerebro. 
Alex es un estudiante muy presumido, el sabe que no le va muy bien con los tests de habilidad mental, pero realmente quiere tener un buen resultado y no le importa como. Encontro una forma de dado la pantalla con todos los puntos tener todas las coordenadas de cada punto con su respectivo color. Lamentablemente Alex no es tan bueno programando como lo es presumiendo, por tanto solicita tu ayuda. A cambio promete ser un poco menos presumido contigo.

\subsection*{Input}

La entrada comienza con un numero N tal que $1 < N \leq 1000$ que representa cuantos puntos hay en la pantalla, luego siguien N lineas cada una con 2 numeros enteros y un String representando las coordenadas x, y del punto y el color respectivamente tal que $0 < x, y \leq 1000$. Se garantiza que no siempre habran mas de 2 puntos por color y que todos los puntos son unicos.

\subsection*{Output}

Debes imprimir cada color existente con la minima distancia entre los puntos de ese color, el numero debe estar aproximado con 2 digitos y el orden de salida debe estar ordenado alfabeticamente en funcion al color.

\datos
