\problem{Tunquña}{tuncuna}

Bob ve a su hermano menor Jack jungando Tunquña. El esta facinando con lo interesante del juego, y decidio jugarlo.

Se pintan una secuencia de cuadrados en el suelo con la ayuda de una tiza, y se le asigna un numero a cada cuadrado(1, 2, 3, 4, ...). Bob esta frente a esos cuadrados. A partir de aqui, el lanza una piedra al primer cuadrado, entonces Bob se mueve a ese cuadrado y recoge la piedra, entonces el vuelve a lanzar la piedra esta vez 2 cuadrados adelante, se mueve a ese cuadrado y recoge la piedra, vuelve a lanzar la piedra pero esta vez 3 cuadrados adelante y asi sucesivamente. ¿Cual es el objetivo del juego?. El objetivo es comprobar si es posible llegar al N-esimo cuadrado con el procedimiento descrito.
Bob es un poco perezoso. El jugara solo si esta seguro que puede llegar al N-esimo cuadrado. Ayuda a Bob a decidir si jugara o no.

\subsection*{Input}

La primera linea contiene un  numero entero T ($1 \leq T \leq 10^5$) que indica el numero de veces que Bob jugara el juego. Cada una de las T siguientes lineas contiene un unico entero N ($1 \leq N \leq 10^{18}$)que denota el N-esimo cuadrado.

\subsection*{Output}

La salida se compone de varias lineas(una linea por cada juego), siguiendo los siguientes criterios: Si Bob es capaz de llegar al N-esimo cuadrado, entonces imprima "Go On Bob"(sin comillas) seguido del numero de movimientos necesarios para llegar al N-esimo cuadrado, ambos separados por un espacio. Si Bob no es capaz de llegar al N-esimo cuadrado, imprima "Better Luck Next Time" (sin comillas).

\subsection*{Explicacion de los casos}
En el primer juego Bob puede saltar al cuadrado 1 luego el solo puede saltar al cuadrado 3, entonces no hay forma de llegar al cuadrado 2.
En el segundo juego como se explico en el primer juego, el primero pude saltar al primer cuadrado, lanzar nuevamente la piedra entonces llegara al 3er cuadrado.

\datos
